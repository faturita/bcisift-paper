\documentclass[12pt]{letter} % Uses 10pt

\usepackage[spanish]{babel}
\usepackage[utf8]{inputenc}
%\usepackage{tgschola}
\usepackage{multirow}
\usepackage{graphicx}
\usepackage{microtype}

\topmargin=-1in    % Make letterhead start about 1 inch from top of page 
\textheight=9in  % text height can be bigger for a longer letter
\oddsidemargin=0pt % leftmargin is 1 inch
\textwidth=6.5in   % textwidth of 6.5in leaves 1 inch for right margin

\begin{document}
	
	\signature{
		\\ Rodrigo Ramele \\
		 }           % name for signature 
	\longindentation=0pt                       % needed to get closing flush left
	\let\raggedleft\raggedright                % needed to get date flush left
	
	
	
	\begin{letter}{ }
		\date{April 4th, 2021}
		
		\begin{flushleft}
		%	M. Juliana Gambini
		\end{flushleft}
		\smallskip\hrule height 2pt
		\begin{flushright}
			\begin{tabular}{rl}
				\multirow{3}{*}[-1.3em]{\includegraphics[width=3.5cm]{derf__64049_1952016_LOGOITBA.jpg}}	\\
				& \small ITBA -- Instituto tecnológico de Buenos Aires\\
				& \small Departamento de Ingeniería Informática\\
				& \small Lavarden 315\\
				& \small  Buenos Aires, Argentina
			\end{tabular}
		\end{flushright} 
		%\vfill % forces letterhead to top of page
		
		\opening{\vspace{-20pt}Dear Editors,} 
		
We are submitting a Full Paper, entitled Capturing Waveforms in Polysomnography which we believe is within the scope of Special Issue on Analysis of 1D Biomedical Signals through AI-based Approaches for Image Processing for the Biomedical Signal Processing and Control Journal.  This article encompass work performed in collaboration between the  CiC Laboratory and the Laboratorio del Sueño of the ITBA University in Buenos Aires, Argentina in the context of research on Electroencephalographic Processing.  This is an outgoing project, which we summarize in the manuscript and apply specifically to Sleep Research.

The work expands a method that we propose to analyze EEG signals based on the structure of the waveforms that are obtained by creating plotting images of the signals.  The method is an extension of the Scale Invariant Feature Transform (SIFT) method used in Computer Vision.  We used the method to analyze a public dataset of Polysomnography (PSG) where we use it to recognize Slow Waves (SW) which are very important in Sleep Research.

We believe this is an interesting work and we think it is very appropriate for this special issue, which emphasize methods that are gounded in 2D image processing and used to analyze 1D signals.
		
Sincerely Yours \\
Rodrigo Ramele
		
		%\encl{Dois formulários 0008/SPT-UFAL;V.2}  				% Enclosures
		
	\end{letter}
	
\end{document}







\end{document}