Dear Editor, 

We are submitting a Full Paper, entitled "Capturing Waveforms in Polysomnography" which we believe is within the scope of "Special Issue on Analysis of 1D Biomedical Signals through AI-based Approaches for Image Processing" for Biomedical Signal Processing and Control.  This article encompass work performed in collaboration between the  CiC Laboratory and the "Laboratorio del Sueño" of the ITBA University in Buenos Aires, Argentina in the context of research on Electroencephalographic Processing.  This is an outgoing project, which we summarize in the manuscript and apply specifically to Sleep Research.

The work expands a method that we propose to analyze EEG signals based on the structure of the waveforms that are obtained by creating plotting images of the signals.  The method is an extension of the Scale Invariant Feature Transform (SIFT) method used in Computer Vision.  We used the method to analyze a public dataset of Polysomnography (PSG) where we use it to recognize Slow Waves (SW) which are very important in Sleep Research.

We believe this is an interesting work and we think it is very appropriate for this special issue, which emphasize methods that are gounded in 2D image processing and used to analyze 1D signals.

Yours truly
Rodrigo Ramele
ITBA University 
Argentina